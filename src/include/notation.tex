
%
% General
%

\DeclareMathOperator*{\argmin}{arg\,min}
\DeclareMathOperator*{\argmax}{arg\,max}
\newcommand*{\dd}{\mathrm{d}}
\newcommand*{\T}{\mathrm{T}}
\newcommand*{\QED}{\hfill $\square$}
\newcommand*{\vecarrow}{}\let\vecarrow\vec
\renewcommand*{\vec}[1]{\bm{\mathrm{#1}}}
\newcommand*{\vx}{\underline{x}}
\newcommand*{\vy}{\underline{y}}
\renewcommand*{\dim}{\textrm{dim}}
\newcommand*{\cuploop}[2]{\overset{#1}{\underset{#2}{\text{\LARGE $\cup$}}}}
\newcommand*{\caploop}[2]{\overset{#1}{\underset{#2}{\text{\LARGE $\cap$}}}}

%
% Numbers
%

\newcommand*{\RR}{\mathbb{R}} % Reals
\newcommand*{\RRpos}{\RR^{+}} % Positiver reals
\newcommand*{\QQ}{\mathbb{Q}} % Rationals
\newcommand*{\ZZ}{\mathbb{Z}} % Whole numbers
\newcommand*{\NN}{\mathbb{N}} % Naturals
\newcommand*{\NNpos}{\NN^{+}} % Positive whole numbers

%
% Disjoint unions
%

% http://tex.stackexchange.com/a/52673/8744
\makeatletter
\def\moverlay{\mathpalette\mov@rlay}
\def\mov@rlay#1#2{\leavevmode\vtop{%
   \baselineskip\z@skip \lineskiplimit-\maxdimen
   \ialign{\hfil$\m@th#1##$\hfil\cr#2\crcr}}}
\newcommand{\charfusion}[3][\mathord]{
    #1{\ifx#1\mathop\vphantom{#2}\fi
        \mathpalette\mov@rlay{#2\cr#3}
      }
    \ifx#1\mathop\expandafter\displaylimits\fi}
\makeatother

\newcommand*{\cupdot}{\charfusion[\mathbin]{\cup}{\cdot}}
\newcommand*{\bigcupdot}{\charfusion[\mathop]{\bigcup}{\cdot}}

%
% Probability
%

\let\Pr\relax
\DeclareMathOperator{\Pr}{\mathbb{P}}
\DeclareMathOperator{\Ex}{\mathbb{E}}

%
% Hipergraphs
%

\newcommand*{\HH}{\mathcal{H}}
\newcommand*{\EE}{\mathcal{E}}
\newcommand*{\GG}{\mathcal{G}}
\newcommand*{\chromatic}{\chi}
\newcommand*{\weakChromatic}{\chi_{\textrm{gyenge}}}
\newcommand*{\strongChromatic}{\chi_{\textrm{erős}}}
\newcommand*{\edgeChromatic}{\chi_{e}}

%
% Sets, Families of sets
%

\renewcommand*{\AA}{\mathcal{A}}
\newcommand*{\FF}{\mathcal{F}}


%%% Local Variables:
%%% mode: latex
%%% TeX-master: "../main"
%%% End:
