\chapter{Hipergráfok és gráfparaméterek}

\begin{dfn}
  A $\HH = (V, \EE)$ pár \emph{hipergráf} (vagy \emph{halmazrendszer}), ahol $V$
  a \emph{csúcsok} halmaza, és $\EE \subseteq 2^V$ a \emph{hiperélek} halmaza.
\end{dfn}

\begin{dfn}
  A $\HH = (V, \EE)$ hipergráf \emph{$r$-uniform}, ha minden $E \in \EE$ hiperél
  pontosan $\lvert E \rvert = r$ csúcsot tartalmaz.
\end{dfn}

Minden $G = (V, E)$ gráf tekinthető $2$-uniform hipergráfnak.

\begin{dfn}
  A $\HH = (V, \EE)$ $r$-uniform hipergráf \emph{$r$-színes}, ha a csúcshalmaza
  partícionálható $V = V_1 \cupdot V_2 \cupdot \ldots \cupdot V_k$ halmazokra
  úgy, hogy minden $E \in \EE$ él minden $V_i$ halmazból pontosan egy csúcsot
  tartalmaz.
\end{dfn}

\begin{dfn}
  Egy $\HH$ hipergráf \emph{gyengén $k$ színnel színezhető}, ha ki lehet színezni a csúcsait $k$ színnel úgy, hogy nincsen egyszínű éle.
  Egy $\HH$ hipergráf \emph{gyenge kromatikus száma} a legkisebb olyan $\weakChromatic(\HH) = k$, amelyre $\HH$ gyengén $k$ színnel színezhető.
\end{dfn}

\begin{dfn}
  Egy $\HH$ hipergráf \emph{erősen $k$ színnel színezhető}, ha ki lehet színezni a csúcsait $k$ színnel úgy, hogy egy élen belül nem tartalmaz két azonos színű csúcsot.
  Egy $\HH$ hipergráf \emph{erős kromatikus száma} a legkisebb olyan $\strongChromatic(\HH) = k$, amelyre $\HH$ erősen $k$ színnel színezhető.
\end{dfn}

\textbf{TODO: klikkgráf}

\begin{dfn}
  Egy $\HH$ hipergráf \emph{$k$ színnel élszínezhető}, ha ki lehet színezni éleit $k$ színnel úgy, hogy az egymást metsző hiperélek nem lehetnek azonos színűek.
  Egy $\HH$ hipergráf \emph{élkromatikus száma} a legkisebb olyan $\edgeChromatic(\HH) = k$, amelyre $\HH$ erősen $k$ színnel színezhető.
\end{dfn}

%%% Local Variables:
%%% mode: latex
%%% TeX-master: "../main"
%%% End:
