\chapter{Hipergráfok és gráfparaméterek}

\begin{dfn}
  A $\HH = (V, \EE)$ pár \emph{hipergráf} (vagy \emph{halmazrendszer}), ahol $V$
  a \emph{csúcsok} halmaza, és $\EE \subseteq 2^V$ a \emph{hiperélek} halmaza.
\end{dfn}

\begin{dfn}
  A $\HH = (V, \EE)$ hipergráf \emph{$r$-uniform}, ha minden $E \in \EE$ hiperél
  pontosan $\lvert E \rvert = r$ csúcsot tartalmaz.
\end{dfn}

Minden $G = (V, E)$ gráf tekinthető $2$-uniform hipergráfnak.

\begin{dfn}
  A $\HH = (V, \EE)$ $r$-uniform hipergráf \emph{$r$-színes}, ha a csúcshalmaza
  partícionálható $V = V_1 \cupdot V_2 \cupdot \ldots \cupdot V_k$ halmazokra
  úgy, hogy minden $E \in \EE$ él minden $V_i$ halmazból pontosan egy csúcsot
  tartalmaz.
\end{dfn}

%%% Local Variables:
%%% mode: latex
%%% TeX-master: "../main"
%%% End:
