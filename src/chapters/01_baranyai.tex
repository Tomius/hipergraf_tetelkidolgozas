\chapter{Baranyai tétele}

\begin{dfn}
  A $\GG$ gráfban egy $\HH$ részgráfot \emph{k-faktor}nak nevezünk, ha $\HH$ k-reguláris (vagyis $\HH$ minden csúcsának k a fokszáma) és feszítő részgráf ($\HH$ tartalmazza $\GG$ összes csúcsát).
\end{dfn}

A teljes párosítás egy 1-faktor.

\begin{dfn}
  Egy gráf éleinek particíóját k-faktorokba \emph{k-faktorizációnak nevezzük}. Ha egy gráfnak létezik \emph{k-faktorizációja}, akkor azt mondjuk, hogy \emph{k-faktorizálható}.
\end{dfn}

Az 1-faktorizációnak fontos szerepe van például sakkversenyek szervezésekor. Ha egy gráfban a csúcspontok a versenyzőket, az élek pedig lebonyolítandó játszmákat jelölik, akkor egy ilyen gráfnak az 1-faktorizációja egy olyan összeállítást ad meg, ahol egy faktor egy fordulónak felel meg, minden fordulóban minden játékos játszik, és semelyik pár sem játszik többször egymás ellen.

\begin{thm}
  Tetszőleges pozitív, páros $n$-re a $K_{n}$, az $n$-csúcsú teljes gráf 1-faktorizálható.
\end{thm}

A tétel legegyszerűbb bizonyítása a következő: $i = 1 , 2 , \dots , n - 1$ esetén az $i$-edik faktor álljon a $(0,~i)$ élből és az $(i-j , i+j) (mod~n - 1)$ élekből, ahol $j = 1 , 2 , \dots , \frac{n}{2} - 1$. A $K_8$ gráf esetén ez a konstrukció \aref{fig:faktorizacio}. ábrán látható.

\includefigure[width=\textwidth]{faktorizacio}{A $K_8$ egy lehetséges egy faktorizációja}

\begin{thm} Baranyai-tétel:
  Minden $r|n$ viszonyt teljesítő $r,n$ párosra a $K^{r}_{n}$, az $r$-uniform $n$-csúcsú teljes hipergráf 1-faktorizálható.
\end{thm}

Egy ezzel ekvivalens tétel:
\begin{thm}
  Minden $r|n$ és $1 \leq k \leq n$ viszonyt teljesítő $r,n,k$ értékekre a $\{1, 2, \dots, k\}$ halmaznak (ezt a halmazt jelölje $[k]$ a későbbiekben) van $M = \binom{n-1}{r-1}$ db olyan partíciója legfeljebb r méretű halmazokra, hogy minden partícióban az $\emptyset$-t is megengedve $m = \frac{n}{r}$ osztály van, és $\forall s \subseteq [k]$ részhalmaz pontosan $\binom{n-k}{r-|s|}$ partícióban fordul elő.
\end{thm}

Megjegyzések:
\begin{itemize}
  \item Ez a Baranyai tételhez hasonló állítást mond ki, de csak az első k elem lehetséges értékeinek particionálásra.
  \item a k=n eset visszaadja a Baranyai tételt, hiszen $\binom{0}{r-|s|}$ értéke egy, ha $s=n$ és nulla, ha $s \neq n$. Azaz az alaphalmaznak létezik $M$ db partíciója $m$ részre, úgy, hogy minden partíció pontosan n elemet tartalmaz (vagyis a partíciók az összes elemet tartalmazzák).
  \item Ennek a tételnek az a jelentősége, hogy indukcióval könnyen bizonyítható, és ez a bizonyítás a Baranyai tételt is igazolja.
\end{itemize}

Bizonyítás (Ekvivalens tétel):
\begin{itemize}
  \item $k=0$-ra $\forall$ minden partícióosztály üres, és $\binom{n}{r}$-szer fordul elő. Ez megegyezik a tétel állításával.

  \item Lemma az indukciós lépéshez:
  Legyen $A = \{a_{i,j}\}, i \in \{1, \dots, t\}, j \in \{1, \dots, h\}$ olyan mátrix, amiben a sor és oszlopösszegek egész számok.

  Ekkor $\exists$ olyan $B = \{b_{i,j}\}, i \in \{1, \dots, t\}, j \in \{1, \dots, h\}$ mátrix, ahol $b_{i,j}~=~\lceil a_{i,j} \rceil$ vagy $b_{i,j}~=~\lfloor a_{i,j} \rfloor$, és minden sor és oszlopösszeg megegyezik az A-belivel.

  Biz: Ha A minden eleme egész, akkor készen vagyunk. Különben induljunk ki egy nem egész elemből. Az egész sorösszeg miatt lesz a sorában legalább egy másik nem egész elem, aminek az oszlopában szintén lesz egy nem egész elem, stb. A véges sok elem miatt előbb-utóbb elérünk egy olyan elemhez, ahol egyszer már jártunk. Az útvonalban felhasznált elemnek lesz olyan részhalmaza, ami egy páros sok elemből álló kört alkot (a vízszintes és függőleges mozgások váltakozása miatt). Legyen $\delta$ az a legkisebb abszolút értékű szám, amit az egyik elemhez hozzá kell adni, hogy egész legyen. A kör mentén felváltva $+\delta$-val és $-\delta$-val módosítsuk az elemeket, úgy, hogy az egészhez legközelebbi elem egésszé váljon. Így legalább eggyel több egész szám lesz a mátrixban, és a sor- és oszlopösszegek nem változnak. Ezt az algoritmust ismételjük, amíg minden elem egész nem lesz.

  \item Indukciós lépés $k$-ra:
  Tekintsük azt a mátrixot, aminek a sorai a $k$-ra vonatkozó állítást teljesítő partíciók, oszlopai $[k]$ részhalmazai és az $i$-edik partíció $s$-hez tartozó oszlopában szereplő érték az $\frac{\binom{n-k-1}{r-|s|-1}}{\binom{n-k}{r-s}} = \frac{r-|s|}{n-k}$, ha $s$ megjelenik az $i$-edik partícióban partícióosztályként (és $s \neq \emptyset$), nulla egyébként. $S = \emptyset$ esetén is ugyanígy járunk el, csak az értéket még felszorozzuk az $\emptyset$ multiplicitásával.

  Erre a mátrixra alkalmazzuk az előbb ismertetett lemmát. Erre egy példa \aref{fig:baranyai} ábrán látható.

  Vegyük észre, hogy kezdetben minden sörösszeg egy. Ha $R$ jelöli az adott sorhoz tartozó partíciót:
  \[\sum_{s \in R} \frac{r-|s|}{n-k} = \frac{\frac{n}{r} \cdot r - \sum\limits_{s \in R} |s|}{n - k} = \frac{n-k}{n-k} = 1\]

  Tehát a kapott egész értékű mátrix minden sorában pontosan egy egyes lesz, aminek a jelentése, hogy az adott sorhoz tartozó partícióban az egyes oszlopának megfelelő partícióosztályt kell bővíteni a $k+1$ nevű elemmel.

  Továbbá minden oszlopban (az $\emptyset$-et kivéve) a nem nulla elemek értéke $\frac{\binom{n-k-1}{r-|s|-1}}{\binom{n-k}{r-s}}$ a mátrix konstrukciója miatt, és nem nulla elemből pontosan $\binom{n-k}{r-s}$ db van (az indukciós feltevés miatt). Ezen számok összege (a szorzatuk) $\binom{n-k-1}{r-|s|-1}$. Vagyis az $k+1$ nevű elem felvétele után a $k'= k+1$ és $s' = s \cup \{k+1\}$-re igaz, hogy $s'$ előfordulásainak száma $\binom{n-k'}{r-|s'|}$

  \includefigure[width=\textwidth]{baranyai}{Az indukciós lépés $k=2$-ről indulva}
\end{itemize}

\QED

%%% Local Variables:
%%% mode: latex
%%% TeX-master: "../main"
%%% End:
