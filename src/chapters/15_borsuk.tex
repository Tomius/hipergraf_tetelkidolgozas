\chapter{ Borsuk ``sejtés'' és Kahn-Kalai-Nilli féle cáfolata}

% \begin{question}
%   Igaz-e, hogy egy $d$ dimenziós halmaz mindig felbontható, legfeljebb $d+1$ az övénél kisebb átmérőjű halmazra.
% \end{question}

$f(d) := min\{k: $tetszőleges $d$ dimenziós halmaz feledhető $k$ db az övénél kisebb átmérőjű halmazzal\}. $f(d) \geq d+1$ adódik a szabályos $d$-dimenziós (tehát $d+1$ csúcsú) szimplexet tekintve.

\medskip

Borsuk-``sejtés'': $f(d) = d+1$.

\begin{thm} Kahn-Kalai:
  Ha $d$ elég nagy, akkor $f(d) > (1+\delta)^{\sqrt{d}}$ alkalmas $\delta > 0$-ra.
\end{thm}

Bizonyítás (A.Nilli):

Ötlet: definiáljunk egy olyan halmazrendszert, ahol a legtávolabbi vektorok (amik meghatározzák az átmérőt) pontosan az egymásra merőleges vektorok. Ennek segítségével definiáljuk az eredeti halmaz egy lefedését olyan halmazrendszerrel, ahol nem engedünk meg merőlegességet (így garantáljuk, hogy az itteni halmazok átmérője kisebb), és ennek az elemszámát becsüljük.

\medskip

$n := 4p, p$ páratlan prím. $\FF := \{x \in \{-1, 1\}^n: x_1=1, |\{i:x_i=-1\}| \equiv 0 \mod{2}\}$

\begin{prop} \label{prop:mod40}
  Ha $\vx \cdot \vy \equiv 0 \mod{p}$ és $\vx, \vy \in \FF$, akkor $\vx \cdot \vy \equiv 0 \mod{4p}$ is.
\end{prop}

Ugyanis:
\begin{prop} \label{prop:mod4pluszminuszize}
  Azon $x_i y_i$-k száma, amire $x_i y_i = 1$ kongruens azon $x_i y_i$-k számával, amire $x_i y_i = -1 \mod{4}$.
\end{prop}

Jelölje $C_{++}$ azon pozíciók ($i$ indexek) számát, ahol $x_i$ és $y_i$ is pozitív, $C_{+-}$ azokat, ahol csak $x_i$ pozitív, $C_{--}$ azokat, ahol egyik se pozitív, $C_{-+}$ pedig ahol csak az $y_i$ pozitív.

Az $\FF$-beli vektorokban a negatív elemek száma páros, így:
\begin{align}
  C_{--} + C_{-+} &= 0 \mod{2} \text{\hspace{5em}| azaz x-ben a -1-ek száma páros} \\
  C_{--} + C_{+-} &= 0 \mod{2} \text{\hspace{5em}| azaz y-ban a -1-ek száma páros} \\
  C_{+-} + 2 \cdot C_{--} + C_{-+} &= 0 \mod{2} \\
  C_{+-} + C_{-+} &= 0 \mod{2} \\
  C_{++} + C_{+-} + C_{--} + C_{-+} &= 4p = 0 \mod{4}  \\
  C_{++} + C_{--} &= C_{+-} + C_{-+} \mod{4}
\end{align}

Ami pont \aref{prop:mod4pluszminuszize}. állítás. Ha ez igaz, akkor viszont $\mod{4}$ ugyanannyi +1-et mint -1-et adunk össze a $\vx \cdot \vy$-ben, tehát ezzel beláttuk \aref{prop:mod40}. állítást is.

\medskip

\begin{obs}
  $\sum\limits_{i=1}^{4p}x_i y_i = -4p$ nem lehet, mert $x_1 = y_1 = 1$ (ha $x, y \in \FF$).
\end{obs}

\medskip

\begin{obs}
  Tehát ha, $\vx \cdot \vy \equiv 0 \mod{p}$, akkor vagy $\vx \cdot \vy = 0$ (azaz $x \bot y$) vagy $\vx \cdot \vy = 4p$ (ami pontosan akkor igaz, ha $x = y$).
\end{obs}

\begin{lem}
  Ha $\GG \subseteq \FF$-ben nincsenek merőleges vektorok, akkor $|\GG| < \sum\limits_{i=0}^{p-1} \binom{n}{i}$
\end{lem}

Ha $\GG$-ben nincsenek merőleges vektorok, akkor csak $\vx = \vy$ esetén lesz $\vx \cdot \vy \equiv 0 \mod{p}$.

Minden $\va \in \FF$-hez definiáljuk a $P_{\va}(\vx)$ függvényt, így:
\[P_{\va}(\vx) = \prod_{i=1}^{p-1} (\va \cdot \vx - i)\]

\begin{obs}
  $P_{\va}(\vx) \not = 0$ pontosan akkor, ha $\va \cdot \vx \equiv 0 \mod{p}$.
\end{obs}

Továbbá legyen $\overline{P_{\va}(\vx)}$ ennek a függvénynek az módosulata, amit úgy kapunk, hogy még hozzávesszük azonosságként, hogy $x_i^2 = 1$ (csak ilyen vektorokra értelmezzük a függvényt).

\begin{prop}
  A $\{\overline{P_{\va}(\vx)}\}_{\va \in \GG}$ polinomok lineárisan függetlenek.
\end{prop}

$\sum\limits_{\va \in \GG} \lambda_{\va} \overline{P_{\va}(\vx)} = 0$ egyenletből $\vx = \vb \in \GG$ helyettesítéssel, és felhasználva, hogy $\va \not = \vb$ esetén $\overline{P_{\va}(\vx)} = 0$ adódik, hogy $\sum\limits_{\va \in \GG} \lambda_{\va} \overline{P_{\va}(\vb)} = \lambda_{\vb} \underbrace{\overline{P_{\vb}(\vb)}}_{\not = 0} = 0 \Rightarrow \lambda_{\vb} = 0$. $\forall \vb$-re $\in \GG$.

Tehát $|G|$ legfeljebb akkora, mint a $P_{\va}(\vx)$ polinomok terének dimenziója (ennél több vektor nem lehetne lineárisan független), ami a lehetséges monomok számával, azaz $\sum\limits_{i=0}^{p-1} \binom{n}{i}$-el becsülhető.

Tétel biz:
$S:= \{\vx \otimes \vx: x \in \FF\}$, ahol a $\otimes$ a Tenzor-szorzat.

\begin{obs}
  $(\vx \otimes \vx) \cdot (\vy \otimes \vy) = \sum\limits_{i,j = 1}^n (x_i x_j)(y_i y_j) = \left(\sum\limits_{i = 1}^n x_i y_i\right) \cdot \left(\sum\limits_{j = 1}^n x_j y_j\right) \\ = (\vx \cdot \vy)^2 \geq 0$.
\end{obs}

Vagyis $(\vx \otimes \vx)$ és $(\vy \otimes \vy)$ pontosan akkor merőleges, ha $x$ merőleges $y$-ra. Továbbá minél kisebb a skalárszorzat értéke, annál távolabb van a két vektor egymástól, szóval a legtávolabbi vektoroknak a skalárszorzata minimális - ami S esetében azt jelenti a legtávolabbi vektorpárok merőlegesek is egymásra.

\medskip

Tehát ha S a felbontandó halmazunk, akkor ha az övénél kisebb átmérőjű halmazokra bontjuk (illetve ilyenekkel fedjük), akkor azokban nem lehetnek merőleges vektorok.

Vagyis ha $S=S_1 \cup \dots \cup S_m$, ahol $\forall i S_i$ átmérője kisebb, mint S átmérője, akkor $|S_i| \leq |\GG| \leq \sum\limits_{i=0}^{p-1} \binom{n}{1}$.

Ekkor $|S| = 2^{n} = \sum\limits_{i=1}^m |S_i| \leq m \cdot  |\GG| \leq m \sum\limits_{i=0}^{p-1} \binom{n}{1} \leq m \cdot p \binom{n}{\frac{n}{4}}$. Ezt átrendezve azt kapjuk, hogy $m \geq \frac{2^{n}}{p \binom{n}{\frac{n}{4}}} \geq (1 + \delta)^n$, alkalmas $\delta > 0$-ra. Ez azért igaz, mert $\binom{n}{\frac{n}{4}} \leq 2^{n h(\frac{1}{4})}$.
