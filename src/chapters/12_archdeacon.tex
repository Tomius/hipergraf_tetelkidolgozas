\chapter{Archdeacon-Hutchinson-Nakamoto-Negami-Ota illetve Mohar-Seymour tétele
nem irányítható felületek négyszögeléseinek kromatikus számáról.}

\begin{thm}
  Ha $\GG$ egy nem irányítható felület olyan négyszögelése, hogy van egy páratlan köre $\GG$-nek, ami mentén a felületet felvágva az irányíthatóvá tehető, akkor $\chi(\GG) \geq 4$.
\end{thm}

Bizonyítás: Tegyük fel, hogy 3 szín elég, és tekintsük egy, az 1, 2, 3 színekkel színezését $\GG$-nek. Irányítsunk minden élt a nagyobb szín felé. Irányítsuk meg a (négyszög)lapokat is úgy (ezt a feltétel miatt lehet), hogy a feltételben szereplő páratlan kör éleinek kivételével minden él két oldalán az irányítás szerinti irányba haladva végig az élen a két végighaladás ellentétes irányú. Ekkor az adott páratlan kör élein pedig sehol sem lesz így.

\medskip

Ez után számoljuk ki az alábbi súlyozott összeget: $D := \sum\limits_{(e,\sigma)} \varphi(e, \sigma)$. Ahol:
\begin{itemize}
  \item $\sigma$ egy lap.
  \item $e$ egy él.
  \item $\varphi(e, \sigma) = +1$, ha $\sigma$ határán a laphoz rendelt irányítás szerint körbehaladva az $e$ élein annak irányításával egyező irányban haladunk.
  \item $\varphi(e, \sigma) = -1$, ha ellenkező irányba haladunk.
\end{itemize}

A D összeget két féleképpen számoljuk:
\begin{enumerate}
  \item Minden él hozzájárulása nulla, ha az él két oldalán lévő lapok irányítása ``konzisztens'' (az egyik oldalon +1-ként, a másik oldalon -1-ként számít). Tehát $2k+1$ ettől eltérő él van valamilyen $k \in \NN$-re. Ezek mindegyikének hozzájárulása pedig $+2$ vagy $-2$. Páratlan sok ilyen van, tehát $D \equiv 2 \mod{4}$.
  \item Egy-egy lap hozzájárulása (a 3-színezés szerinti élirányítás miatt) mindig nulla, hiszen biztosan lesz két szemközti, ugyanolyan színű csúcs - emiatt a két vízszintes és két függőleges él irányítása megegyezik, és ezek ellentétes előjellel számítanak. Így $D = 0$.
\end{enumerate}

Tehát a 3-színezés esetén $D \equiv 2 (\textrm{mod}\ 4)$ és $D = 0$ \Lightning. Vagyis nem lehet 3 színnel jól színezni.

\begin{thm} (Speciális esete az előző tételnek) Youngs tétel:
  A projektív sík négyszögelése sohasem 3-kromatikus: ha nem páros gráf, akkor $\chi \geq 4$.
\end{thm}
