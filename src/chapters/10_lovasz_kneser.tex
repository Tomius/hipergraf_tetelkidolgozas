\chapter{Lovász-Kneser tétel és Greene féle bizonyítása. Dolnyikov tétele és annak magyarázata, hogy miért következik belőle a Lovász-Kneser tétel.}

\begin{notation}
  $S^d: d+1$ dimenziós gömbnek a $d$ dimenziós felszíne.
\end{notation}

\begin{thm} Borsuk-Ulan tétel:
  Ha $f: S^d \rightarrow \RR^d$ folytonos függvény, akkor $\exists x \in S^d$, amire $f(x) = f(-x)$.
\end{thm}

\begin{dfn}
  Egy $U$ halmaz nyílt, ha $\forall x \in U$-nak $\exists$ olyan kis környezete, ami teljesen $U$-ban van. Egy halmaz zárt, ha a komplementere nyílt.
\end{dfn}

\begin{thm} Lusternik-Schnirel'mann tétel:
  \begin{enumerate}
    \item Legyenek $A_1, A_2, \dots, A_m \subseteq S^d$ zárt halmazok, melyekre
    $\cuploop{m}{i=1} A_i = S^d$ és $\forall i, x \in A_i \Rightarrow -x \not \in A_i$. Ekkor $m \geq d + 2$.
    \item ugyanez, ha $A_i$ nyílt.
    \item ugyanez, ha $A_i$ nyílt vagy zárt.
  \end{enumerate}
\end{thm}

Bizonyítás (zárt eset, Borsuk-Ulan tétel alapján):
Tekintsük $S^d$-nek egy $A_1, \dots, A_{d+1}$ halmazokkal való fedését, ahol $\forall A_i$ zárt $(\cuploop{d+1}{i=1} A_i = S^d)$.

$f: x \rightarrow (\text{dist}(x, A_1), \text{dist}(x, A_2), \dots, \text{dist}(x, A_d))$.

Ez egy folytonos $S^d \rightarrow \RR^d$ függvény $\underbrace{\Rightarrow}_{\text{BU tétel}} \exists x_0 \in S^d: f(x) = f(-x)$.

Ha $\exists i$, hogy $x_0 i.$ koordinátája nulla, akkor $x_0, -x_0 \in A_i$. Ha $\not \exists$ ilyen $i$, akkor $x_0, -x_0 \in A_{d+1}$. Vagyis mindenképpen van egy antipodális pontpárt tartalmazó $A_i$.
\QED

\begin{dfn}
  $n, k \in \NN, k < \frac{n}{2}$ esetén az $n, k$ paraméterű $KG(n, k)$ Kneser gráf a következő:

  \begin{flalign}
    V(KG(n,k)) &= \binom{[n]}{k} &\\
    E(KG(n,k)) &= \{AB: A,B \in \binom{[n]}{k}, A \cap B = \emptyset\}
  \end{flalign}
\end{dfn}

\begin{thm} Lovász-Kneser tétel:
  $\chi(KG(n, k)) = n - 2k + 2$
\end{thm}

Bizonyítás (Greene):
Tekintsük $S^{n - 2k + 1}$-et és rajta $n$ pontot általános helyzetben, azaz semelyik $n - 2k + 2$ nem esik közös főkörre.

Vegyük $KG(n, k)$ egy jó színezését $m$ színnel, és definiáljuk az $A_i, \dots, A_m \subseteq S^{n-2k+1}$ nyílt halmazokat úgy, hogy $x \in A_i$, ha az $x$ középpontú nyílt félgömb tartalmaz olyan $k$ pontot, hogy az azoknak megfelelő $K(n, k)$-beli csúcspont az adott színezésben $i$ színű.

\medskip

$\forall A_i$ nyílt, és nem tartalmaz antipodális pontpárt, hiszen ha $x \in A_i$ és $-x \in A_i$, akkor $H(x)$ és $H(-x)$ tartalmaz két diszjunkt $k$-ast, ami $KG(n,k)$-ben két összekötött csúcspont, és mindkettő $i$ színű \Lightning.

\medskip

$B := S^{n-2k+1} \backslash \cuploop{n}{i=1} A_i$, \hspace{1em} $B$ zárt.

\medskip

$x \in B \Rightarrow -x \not \in B$, mert ha $x \in B$, akkor $H(x)$ legfeljebb $k-1$ pontot tartalmaz (ha lenne rajta $k$ pont, akkor $x$ valamelyik $A_i$-nek része lenne), és ugyanígy ha $-x \in B$, akkor $H(-x)$ legfeljebb $k-1$ pontot tartalmaz. Emiatt ha $x, -x \in B$, akkor a $H(x)$ és $H(-x)$-et elválasztó főkörön legalább $n-2(k-1)=n-2k+2$ pont van. \Lightning.

\medskip

A Lusternik-Schnirel'mann tétel (nyílt és zárt vegyes változata) miatt:

\begin{align}
m + 1 &\geq n - 2k + 1 + 2 \\
m &\geq n - 2k + 2
\end{align}

\QED

\begin{dfn}
  Legyen $\HH=(V, E)$ egy hipergráf. $cd_2(\HH) = \min\{|u| : u \subseteq V, \chi(\HH \backslash u) \leq 2\}$
\end{dfn}

\begin{dfn}
  Egy $\FF$ halmazrendszerhez rendelt általánosított Kneser-gráf, $KG(\FF)$ a következő:

  \begin{flalign}
    V(KG(\FF)) &= \FF &\\
    E(KG(\FF)) &= \{AB: A,B \in \FF, A \cap B = \emptyset\}
  \end{flalign}
\end{dfn}

\begin{thm} Dolnyikov tétel:
  $\forall \FF \subseteq 2^{[n]}: \chi(KG(\FF)) = cd_2(\FF)$
\end{thm}

Ez általánosítása a Lovász-Knézer tételnek:
$\FF := \binom{[n]}{k}, cd_2(\FF) = n-2(k-1) = n-2k+2$ (ugyanis két megmaradó színosztály egyikében sem lehet maradhat $k$ pont, szóval kevesebb pont elhagyása nem elég, de ha mindkét színosztályban $k-1$ pont marad, akkor a gráf már kettő kromatikus). Vagyis $\chi(KG(\FF)) = cd_2(\FF) = n-2k+2$.
