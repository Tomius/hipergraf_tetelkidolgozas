\chapter{Lovász-Kneser tétel és Greene féle bizonyítása. Dolnyikov tétele és annak magyarázata, hogy miért következik belőle a Lovász-Kneser tétel.}

\begin{notation}
  $S^d: d$ dimenziós gömb felszíne.
\end{notation}

\begin{thm} Borsuk-Ulan tétel:
  Ha $f: S^d \rightarrow \RR^d$ folytonos függvény, akkor $\exists x \in S^d$, amire $f(x) = f(-x)$.
\end{thm}

\begin{dfn}
  Egy $U$ halmaz nyílt, ha $\forall x \in U$-nak $\exists$ olyan kis környezete, ami teljesen $U$-ban van. Egy halmaz zárt, ha a komplementere nyílt.
\end{dfn}

\begin{thm} Lusternik-Schnirel'mann tétel:
  \begin{enumerate}
    \item Legyenek $A_1, A_2, \dots, A_m \subseteq S^d$ zárt halmazok, melyekre
    $\overset{m}{\underset{i=1}{\text{\LARGE $\cup$}}} A_i = S^d$ és $\forall i, x \in A_i \Rightarrow -x \not \in A_i$. Ekkor $m \geq d + 2$.
    \item ugyanez, ha $A_i$ nyílt
    \item ugyanez, ha $A_i$ nyílt vagy zárt
  \end{enumerate}
\end{thm}

Bizonyítás (zárt eset, Borsuk-Ulan tétel alapján):
Tekintsük $S^d$-nek egy $A_1, \dots, A_{d+1}$ halmazokkal való fedését, ahol $\forall A_i$ zárt $(\overset{d+1}{\underset{i=1}{\text{\LARGE $\cup$}}} A_i = S^d)$.

$f: x \rightarrow (\text{dist}(x, A_1), \text{dist}(x, A_2), \dots, \text{dist}(x, A_d))$.

Ez egy folytonos $S^d \rightarrow \RR^d$ függvény $\underbrace{\Rightarrow}_{\text{BU tétel}} \exists x_0 \in S^d: f(x) = f(-x)$.

Ha $\exists i$, hogy $x_0 i.$ koordinátája nulla, akkor $x_0, -x_0 \in A_i$. Ha $\not \exists$ ilyen $i$, akkor $x_0, -x_0 \in A_{d+1}$. Vagyis mindenképpen van egy antipodális pontpárt tartalmazó $A_i$.
\QED

\begin{dfn}
  $n, k \in \NN, k < \frac{n}{2}$ esetén az $n, k$ paraméterű $KG(n, k)$ Kneser gráf a következő:

  \begin{flalign}
    V(KG(n,k)) &= \binom{[n]}{k} &\\
    E(KG(n,k)) &= \{AB: A,B \in \binom{[n]}{k}, A \cap B = \emptyset\}
  \end{flalign}
\end{dfn}

\begin{thm} Lovász-Kneser tétel:
  $\chi(KG(n, k)) = n - 2k + 2$
\end{thm}
