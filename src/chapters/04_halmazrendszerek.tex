\chapter{Halmazrendszerek és árnyékaik}

\begin{dfn} Egy halmazrendszer árnyéka:
  $\partial \FF = \{B \in \binom{[n]}{r-1}: \exists F \in \FF, B \subseteq F\}$
\end{dfn}

Probléma: Ha $\FF \subseteq \binom{[n]}{r}$ és egy fix $m$-re $|\FF|=m$, akkor hogyan kell megválasztani $\FF$ elemeit, hogy $|\partial \FF|$ minimális legyen?

\vspace{1em}

Példa $n=8, r=3$ esetén olyan $\FF$-ekre, ahol $|\partial \FF|$ minimális:
\begin{itemize}
  \item $\left\{ \begin{aligned}
    m &= 2 \\
    \FF &= \{123, 234\} \hspace{2em} \left(\text{ami egy jelölés erre:} \{\{1,2,3\}, \{2,3,4\}\}\right) \\
    \partial \FF &= \{12, 23, 13, 24, 34\}
  \end{aligned} \right.$

  \item $\left\{ \begin{aligned}
    m &= 3 \\
    \FF &= \{123, 234, 124\} \\
    \partial \FF &= \{12, 23, 13, 24, 34, 14\}
  \end{aligned} \right.$

  \item $\left\{ \begin{aligned}
    m &= 4 \\
    \FF &= \{123, 234, 124, 134\} \\
    \partial \FF &= \{12, 23, 13, 24, 34, 14\}
  \end{aligned} \right.$

  \item $\left\{ \begin{aligned}
    m &= 5 \\
    \FF &= \{123, 234, 124, 134, 125\} \\
    \partial \FF &= \{12, 23, 13, 24, 34, 14, 15, 25\}
  \end{aligned} \right.$

  \item $\left\{ \begin{aligned}
    m &= 6 \\
    \FF &= \{123, 234, 124, 134, 125, 235\} \\
    \partial \FF &= \{12, 23, 13, 24, 34, 14, 15, 25, 35\}
  \end{aligned} \right.$

\end{itemize}

\begin{obs}
  Az előbbi példa alapján kézenfekvőnek tűnik az a sejtés, hogy a $\binom{[n]}{r}$ halmazoknak létezik olyan sorrendje, hogy a sorrend első $m$ elemének az árnyéka minimális. A későbbiekben látni fogjuk, hogy ez a sejtés igaz.
\end{obs}

Megjegyzés: A halmazok közötti sorrend értelmezéséhez azt feltételezzük, hogy a halmazok elemei egy halmazon belül balról-jobbra növekvő sorrendben vannak rendezve. Tehát például a $\{3, 5, 1\}$ halmazban az elemek belső sorrendjének az $1, 3, 5$ sorozatot értjük.

\vspace{1em}

\noindent A $\binom{[n]}{r}$ nevezetes sorrendjei közül kettő:
\begin{itemize}
  \item lexikografikus:\hphantom{ko} $\{a_1, a_2, ..., a_r\} < \{b_1, b_2, ..., b_r\}$, ha $\exists i: a_i < b_i$ és $\forall j < i: a_j = b_j$.

  Pl: $123, 124, 125, 126, 127, 128, 134, 135, 136, 137, 138, 145, 146, 147, 148, 156, \dots$

  \item kolexikografikus: $\{a_1, a_2, ..., a_r\} < \{b_1, b_2, ..., b_r\}$, ha $\exists i: a_i < b_i$ és $\forall j > i: a_j = b_j$.

  Pl: $123, 124, 134, 234, 125, 135, 235, 145, 245, 345, 126, 136, 236, 146, 246, 346, \dots$
\end{itemize}

A két sorrend közötti különbség, hogy a sorrendben a következő elem előállításhoz a lexikografikus rendezés a legjobboldali, azaz a legnagyobb elemet fogja növeli (ha az még nem érte el a maximális értékét), míg a kolexikografikus sorrend a legbaloldalibbat változtatja, azaz a legkisebb elemet, amit lehet növelni.

Ebből fakadóan például a $\binom{[\infty]}{r}$ halmazok felsorolásánál a lexikografikus sorrend csak az utolsó számjegyet fogja növelni a végtelenségig. Ezzel szemben a kolexikografikus rendezés a $k$ nevű elemet csak akkor veszi be a halmazba, ha a $\binom{[k-1]}{r}$ összes halmazát már felsorolta, tehát már a $k-1$ a legkisebb elem, amit növelni tud új halmaz generáláshoz. Így a kolexikografikus rendezés a $\binom{[\infty]}{r}$ összes halmazát fel tudja sorolni, amit a lexikografikus rendezés erre nem képes.

A kolexikografikus rendezésnek az a tulajdonsága, hogy a nagy elemeket a ``lehető legkésőbb'' használja csak fel, amikor a kisebb elemekből több halmazt nem tud már generálni, érdekessé teszi ezt a sorozatot az árnyék minimalizálás szempontjából.

\begin{thm}
  $\forall m, r \in \NN \exists m$-nek $r$-binomiális felírása:
  \[m = \binom{m_r}{r} + \binom{m_{r-1}}{r-1} + \dots \binom{m_s}{s}\], ahol $m_r > m_{r-1} > \dots > m_s$(, az s értéke m-től és r-től függ), és ez a felírás egyértelmű.
\end{thm}

\begin{samepage} % TODO: ez miert nem mukodik?
  Biz: Ilyen felírást kapunk az alábbi módon: \\
  $m_r = max~\{k: \binom{k}{r} \leq m\}$ \\
  $m_{r-1} = max~\{k: \binom{k}{r-1} \leq m - \binom{m_r}{r}\}$ \\
  \vdots \\
  $m_j = max~\{k: \binom{k}{r-1} \leq \underbrace{m - \binom{m_r}{r} - \binom{m_{r-1}}{r-1} - \dots - \binom{m_{j+1}}{j+1}}_{d_{j+1}}\}$
\end{samepage}

Végül elérünk $d_s = 0$-hoz, hiszen ha $s=1$, akkor minden $d_s = m_s$ szám felírható $\binom{m_s}{s} = \binom{m_s}{1} = m_s$ alakban. Így kapunk egy felírást $m = \binom{m_r}{r} + \dots + \binom{m_s}{s}$ alakban, kell még, hogy erre $m_r > \dots > m_s$ teljesül.

Tegyük fel, hogy valamely j-re $m_j \geq m_{j+1}$. Ekkor $\binom{m_j}{j} \geq \binom{m_{j+1}}{j}$, ami átrendezhető arra az alakra, hogy $\underbrace{\binom{m_{j+1}}{j+1} + \binom{m_j}{j}}_{\geq d_{j+2}} \geq \binom{m_{j+1}}{j+1} + \binom{m_{j+1}}{j} = \binom{m_{j+1} + 1}{j+1}$ \Lightning. Ez azért ellentmondás, mert így $\binom{m_{j+1}+1}{j+1} \leq d_{j+2}$ miatt $m_{j+1}$ helyett választhattuk volna $(m_{j+1} + 1)$-et is. Tehát $m_r > \dots > m_s$ teljesül, a felírás létezik.

Kell még, hogy a felírás egyértelmű. Indirekt tegyük fel, hogy nem az, vagyis $\exists$ két ilyen felírása egy $m$ számnak: $m = \binom{m_r}{r} + \dots + \binom{m_s}{s}$, ahol $m_r > \dots > m_s$ és $m = \binom{m_r'}{r} + \dots + \binom{m_s}{s}$, ahol $m_r' > \dots > m_s'$. Legyen $j$ a legnagyobb index, amire $m_j = m_j'$. Vagyis $d_{j+1}$-nek van két felírása, ahol az első tagok különböznek: $\binom{m_j}{j} + \binom{m_{j-1}}{j-1} + \dots + \binom{m_s}{s} = \overbrace{m - \binom{m_r}{r} - \dots - \binom{m_{j+1}}{j+1}}^{d_{j+1}} = m - \binom{m_r'}{r} - \dots - \binom{m_{j+1}''}{j+1} = \underbrace{\binom{m_j'}{j} + \binom{m_{j-1}'}{j-1} + \dots + \binom{m_s'}{s}}_{A}$.

WLOG (Without Loss Of Generality) tegyük fel, hogy $m_j > m_j'$. Ekkor $m_j' \leq m_j-1$, vagyis $A \leq \underbrace{\binom{m_j-1}{j} + \binom{m_j-2}{j-1} + \dots + \binom{m_j-j}{1}}_{B} < 1+B = \\ = \underbrace{1 + \binom{m_j-j}{1}}_{\binom{m_j-(j-1)}{1}} + \binom{m_j-(j-1)}{2} + \dots + \binom{m_j-1}{j} = \underbrace{\binom{m_j-(j-1)}{1} + \binom{m_j-(j-1)}{2}}_{\binom{m_j-(j-2)}{2}} + \dots + \binom{m_j-1}{j} = \binom{m_j}{j}$.

Vagyis $A = \binom{m_j}{j} + \binom{m_{j-1}}{j-1} + \dots + \binom{m_s}{s} < \binom{m_j}{j}$, ami ellentmondás \Lightning. Tehát a felírás egyértelmű.

\QED

Jelölés: \\
$b^{(r)}(m_r, m_{r-1}, \dots, m_s) = \binom{m_r}{r} + \binom{m_{r-1}}{r-1} + \dots \binom{m_s}{s}$\\
$\AA \oplus \{a_{i1}, a_{i2}, \dots, a_{ih}\}: \{A \cup \{a_{i1}, a_{i2}, \dots, a_{ih}\}: A \in \AA\} $\\
$\AA \ominus \{a_{i1}, a_{i2}, \dots, a_{ih}\}: \{A \backslash \{a_{i1}, a_{i2}, \dots, a_{ih}\}: A \in \AA\} $\\
$B^{(r)}(m_r, m_{r-1}, \dots, m_s) =$ a következő halmazrendszer: \\
$\binom{[m_r]}{r} \cup \{\binom{[m_{r-1}]}{r-1} \oplus \{m_r + 1\}\} \cup \{\binom{[m_{r-2}]}{r-2} \oplus \{m_r + 1, m_{r-1} + 1\}\} \cup \dots$

\bigbreak

Észrevételek:
\begin{itemize}
\item A $B^{(r)}(m_r, m_{r-1}, \dots, m_s)$ az első $b^{(r)}(m_r, m_{r-1}, \dots, m_s)$ db $r$ elemű részhalmaza $\NN$-nek a kolexikografikus rendezés szerint.
\item $b^{(r)}(m_r, m_{r-1}, \dots, m_s) = b^{(r)}(m_r-1, m_{r-1}-1, \dots, m_s-1) + b^{(r-1)}(m_r-1, m_{r-1}-1, \dots, m_s-1)$. Ez egyszerűen abból következik, hogy minden $j$-re $\binom{m_j}{j} = \binom{m_j-1}{j} + \binom{m_j-1}{j-1}$
\item $\partial B^{(r)}(m_r, \dots, m_s) = B^{(r-1)}(m_r, \dots, m_s)$
\end{itemize}

\begin{thm} Kruskal-Katona tétel:

  Legyen $r \geq 1$ és $\AA \subseteq \binom{\NN}{r}$. Ekkor $\partial \AA \geq \partial^{(r)}(|\AA|)$, ahol $\partial^{(r)}(m)$ a kolexikografikus első m db r elemű halmaz árnyéka.
\end{thm}

Biz~\footnote{Egy hasonló, de picivel egyszerűbb bizonyítás: \\ \url{http://www.sciencedirect.com/science/article/pii/0012365X84901936}
}: Először definiáljuk az alábbi, balra tolásnak (shiftnek) nevezett operátort halmazokra és halmazrendszerekre:

Egy $A \in \binom{\NN}{k}$ halmaz $R_{ij}$ eltoltja:
\[R_{ij}(A) = \left\{
\begin{array}{lr}
  A \backslash \{j\} \cup \{i\}, & \text{ha } i \not \in A, j \in A  \\
  A, & \text{egyébként}
\end{array}\right\}\]

Egy $\AA \subseteq \binom{\NN}{k}$ halmazrendszer $R_{ij}$ eltoltja:
\[ R_{ij}(\AA) = \{R_{ij}(A): A \in \AA\} \cup \{A: R_{ij}(A) \in \AA, A \in \AA\}.\]

Vegyük észre, hogy $|R_{ij}(\AA)| = |\AA|$.

\begin{lem}
  A $\binom{N}{r}$ árnyéke nem nő a balra tolás által, azaz ha $1 \leq i < j$, akkor $|\partial(R_{ij}(\AA))| \leq |\partial\AA|$.
\end{lem}

Biz:
Vegyük észre, hogy a lemma állításával ekvivalensek az alábbi állítások is:
\begin{itemize}
  \item $|\partial(R_{ij}(\AA)) \backslash \partial\AA| \leq |\partial\AA \backslash \partial(R_{ij}(\AA))|$, mert ha a lemmában az eltolt halmaz nem nagyobb, akkor mindkét halmazból a közös elemeket kivonva az eltolt halmaz továbbra se lesz nagyobb.
  \item $|R_{ij}(\partial R_{ij}(\AA)) \backslash \partial\AA| \leq |\partial A \backslash \partial R_{ij}(\AA)|$, mert az előző egyenlőtlenségben a bal oldalra egy újabb eltolás operátort használva a halmaz elemszáma nem változik meg.
\end{itemize}

Ezeket a megfigyeléseket felhasználva azt fogjuk belátni, hogy
\[R_{ij}(\partial R_{ij}(\AA)) \backslash \partial\AA \subseteq \partial A \backslash \partial R_{ij}(\AA).\]

W.l.o.g: $i:=1, j:=2$. Jelölés: $C \in \binom{\NN}{r-2}, 1,2 \in C$ és $k \in \NN \backslash \{1, 2\}$, illetve $kC$-nek hívjuk azt a halmazt, hogy $\{k\} \cup C$.

Legyen $B \in R_{21}(\partial R_{12} \backslash \partial \AA)$. Ekkor $1 \not \in B, 2 \in B$, azaz a $B$ halmaz $2C$ alakú. Ekkor vagy $1C \in \partial
 R_{12}(\AA) \backslash \partial \AA$, tehát $2C \in \partial \AA$, vagy $2C \in \partial R_{12} (\AA) \backslash \partial \AA$. Az utóbbi viszont nem lehet, mert ha a $\partial R_{12} (\AA)$ egyik halmazában szerepel a $2$ elem, akkor az a $\partial \AA$ megfelelő halmazában is szerepelne, tehát a különbségképzés után már nem szerepelne az eredményben. Tehát $B \in \partial \AA$.

Kell még, hogy $B \not \in \partial R_{12}(\AA)$. Ezt úgy mutatjuk meg, hogy belátjuk, ha $B \in \partial R_{12}(\AA)$, akkor $B \not \in R_{21}(\partial R_{12}(\AA) \backslash \partial \AA)$, ami \Lightning. Tegyük fel tehát, hogy $B \in \partial R_{12}(\AA) \Rightarrow 12C \in R_{12}(\AA)$ vagy $2Ck \in R_{12}(\AA)$. Mindkettő ellentmondás, mert: \[\left.
\begin{array}{l}
  12C \in R_{12}(\AA) \Rightarrow 12C \in \AA \\
  2Ck \in R_{12}(\AA) \Rightarrow 1Ck \in \AA
\end{array}\right\} \Rightarrow 1C \in \partial \AA \Rightarrow 1C \not \in \partial R_{12}(\AA) \backslash \partial \AA, \text{\Lightning.}\]

\QED

A Kruskal-Katona tétel bizonyítása: A lemma alapján feltehető, hogy $\AA$ balra tolt, azaz $\forall 1 \leq i < j$-re $R_{ij}(\AA) = \AA$. Bontsuk $\AA$-t két részre, így:
\begin{itemize}
  \item $\AA_0 = \{A \in \AA: 1 \not \in A\}$,
  \item $\AA_1 = \{A \in \AA: 1 \in A\}$,
  \item $\AA = \AA_0 \cup \AA_1$.
\end{itemize}

Ekkor a balra toltság miatt $\partial \AA_0 \subseteq \AA_1 \ominus \{1\}$, illetve $\partial \AA_1 = (\AA_1 \ominus \{1\}) \cup \partial(\AA_1 \ominus \{1\}) \oplus \{1\}$. Alkalmazzunk dupla indukciót $m$-re és $r$-re. A kezdő elemek triviálisan igaz a Kruskal-Katona tétel állítása:
\begin{itemize}
  \item $m=1$ és tetszőleges $r$ \checkmark
  \item $r=1$ és tetszőleges $m$ \checkmark
\end{itemize}

Indukciós feltevés, hogy $1, 2, \dots, r-1$ és tetszőleges m esetén \checkmark, illetve $m-1$ is \checkmark, $1,2,\dots,r$-re.

\bigbreak

Állítás: $|\AA_1| \geq b^{(r-1)}(m_r-1,\dots, m_s-1)$, ahol $m_r, \dots, m_s$ az m binomiális felírásából származnak, azaz $|\AA| = m =  b^{(r)}(m_r,\dots, m_s)$.

Tegyük fel indirekt, hogy $|\AA_1| < b^{(r-1)}(m_r-1,\dots, m_s-1).$ Ekkor $|\AA_0| = |\AA| - |\AA_1| > b^{(r)}(m_r,\dots, m_s) - b^{(r-1)}(m_r-1,\dots, m_s-1) = b^{(r)}(m_r-1,\dots, m_s-1) \underbrace{\Rightarrow}_{\text{ind. felt.}} |\partial \AA_0| \geq b^{(r-1)}(m_r-1,\dots, m_s-1) > |\AA_1|$, viszont $|\partial \AA_0| < |\AA_1|$, \Lightning, tehát az állítás igaz.

\bigbreak

Mindebből $|\partial \AA| = |\partial \AA_0 \cup \partial \AA_1| = |\partial \AA_1| = |\AA_1| + |\partial(\AA_1 \ominus \{1\})| \underbrace{\geq}_{\text{áll + ind. felt.}} b^{(r-1)}(m_r-1,\dots, m_s-1) + b^{(r-2)}(m_r-1,\dots, m_s-1) =
b^{(r-1)}(m_r,\dots, m_s) = \partial^{(r)}(m)$.

\QED
