\chapter{Halmazrendszerek és árnyékaik}

\begin{dfn} Egy halmazrendszer árnyéka:
  $\partial \FF = \{B \in \binom{[n]}{r-1}: \exists F \in \FF, B \subseteq F\}$
\end{dfn}

Probléma: Ha $\FF \subseteq \binom{[n]}{r}$ és egy fix $m$-re $|\FF|=m$, akkor hogyan kell megválasztani $\FF$ elemeit, hogy $|\partial \FF|$ minimális legyen?

\vspace{1em}

Példa $n=8, r=3$ esetén olyan $\FF$-ekre, ahol $|\partial \FF|$ minimális:
\begin{itemize}
  \item $\left\{ \begin{aligned}
    m &= 2 \\
    \FF &= \{123, 234\} \hspace{2em} \left(\text{ami egy jelölés erre:} \{\{1,2,3\}, \{2,3,4\}\}\right) \\
    \partial \FF &= \{12, 23, 13, 24, 34\}
  \end{aligned} \right.$

  \item $\left\{ \begin{aligned}
    m &= 3 \\
    \FF &= \{123, 234, 124\} \\
    \partial \FF &= \{12, 23, 13, 24, 34, 14\}
  \end{aligned} \right.$

  \item $\left\{ \begin{aligned}
    m &= 4 \\
    \FF &= \{123, 234, 124, 134\} \\
    \partial \FF &= \{12, 23, 13, 24, 34, 14\}
  \end{aligned} \right.$

  \item $\left\{ \begin{aligned}
    m &= 5 \\
    \FF &= \{123, 234, 124, 134, 125\} \\
    \partial \FF &= \{12, 23, 13, 24, 34, 14, 15, 25\}
  \end{aligned} \right.$

  \item $\left\{ \begin{aligned}
    m &= 6 \\
    \FF &= \{123, 234, 124, 134, 125, 235\} \\
    \partial \FF &= \{12, 23, 13, 24, 34, 14, 15, 25, 35\}
  \end{aligned} \right.$

\end{itemize}

\begin{obs}
  Az előbbi példa alapján kézenfekvőnek tűnik az a sejtés, hogy a $\binom{[n]}{r}$ halmazoknak létezik olyan sorrendje, hogy a sorrend első $m$ elemének az árnyéka minimális. A későbbiekben látni fogjuk, hogy ez a sejtés igaz.
\end{obs}

Megjegyzés: A halmazok közötti sorrend értelmezéséhez azt feltételezzük, hogy a halmazok elemei egy halmazon belül balról-jobbra növekvő sorrendben vannak rendezve. Tehát például a $\{3, 5, 1\}$ halmazban az elemek belső sorrendjének az $1, 3, 5$ sorozatot értjük.

\vspace{1em}

\noindent A $\binom{[n]}{r}$ nevezetes sorrendjei közül kettő:
\begin{itemize}
  \item lexikografikus:\hphantom{ko} $\{a_1, a_2, ..., a_r\} < \{b_1, b_2, ..., b_r\}$, ha $\exists i: a_i < b_i$ és $\forall j < i: a_j = b_j$.

  Pl: $123, 124, 125, 126, 127, 128, 134, 135, 136, 137, 138, 145, 146, 147, 148, 156, \dots$

  \item kolexikografikus: $\{a_1, a_2, ..., a_r\} < \{b_1, b_2, ..., b_r\}$, ha $\exists i: a_i < b_i$ és $\forall j > i: a_j = b_j$.

  Pl: $123, 124, 134, 234, 125, 135, 235, 145, 245, 345, 126, 136, 236, 146, 246, 346, \dots$
\end{itemize}

A két sorrend közötti különbség, hogy a sorrendben a következő elem előállításhoz a lexikografikus rendezés a legjobboldali, azaz a legnagyobb elemet fogja növeli (ha az még nem érte el a maximális értékét), míg a kolexikografikus sorrend a legbaloldalibbat változtatja, azaz a legkisebb elemet, amit lehet növelni.

Ebből fakadóan például a $\binom{[\infty]}{r}$ halmazok felsorolásánál a lexikografikus sorrend csak az utolsó számjegyet fogja növelni a végtelenségig. Ezzel szemben a kolexikografikus rendezés a $k$ nevű elemet csak akkor veszi be a halmazba, ha a $\binom{[k-1]}{r}$ összes halmazát már felsorolta, tehát már a $k-1$ a legkisebb elem, amit növelni tud új halmaz generáláshoz. Így a kolexikografikus rendezés a $\binom{[\infty]}{r}$ összes halmazát fel tudja sorolni, amit a lexikografikus rendezés nem tud.

A kolexikografikus rendezésnek az a tulajdonsága, hogy a nagy elemeket a ``lehető legkésőbb'' használja csak fel, amikor a kisebb elemekből több halmazt nem tud már generálni, érdekessé teszi ezt a sorozatot az árnyék minimalizálás szempontjából.
