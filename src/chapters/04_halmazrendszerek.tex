\chapter{Halmazrendszerek és árnyékaik}

\begin{dfn} Egy halmazrendszer árnyéka:
  $\partial \FF = \{B \in \binom{[n]}{r-1}: \exists F \in \FF, B \subseteq F\}$
\end{dfn}

Probléma: Ha $\FF \subseteq \binom{[n]}{r}$ és egy fix $m$-re $|\FF|=m$, akkor hogyan kell megválasztani $\FF$ elemeit, hogy $|\partial \FF|$ minimális legyen?

\vspace{1em}

Példa $n=8, r=3$ esetén olyan $\FF$-ekre, ahol $|\partial \FF|$ minimális:
\begin{itemize}
  \item $\left\{ \begin{aligned}
    m &= 2 \\
    \FF &= \{123, 234\} \hspace{2em} \left(\text{ami egy jelölés erre:} \{\{1,2,3\}, \{2,3,4\}\}\right) \\
    \partial \FF &= \{12, 23, 13, 24, 34\}
  \end{aligned} \right.$

  \item $\left\{ \begin{aligned}
    m &= 3 \\
    \FF &= \{123, 234, 124\} \\
    \partial \FF &= \{12, 23, 13, 24, 34, 14\}
  \end{aligned} \right.$

  \item $\left\{ \begin{aligned}
    m &= 4 \\
    \FF &= \{123, 234, 124, 134\} \\
    \partial \FF &= \{12, 23, 13, 24, 34, 14\}
  \end{aligned} \right.$

  \item $\left\{ \begin{aligned}
    m &= 5 \\
    \FF &= \{123, 234, 124, 134, 125\} \\
    \partial \FF &= \{12, 23, 13, 24, 34, 14, 15, 25\}
  \end{aligned} \right.$

  \item $\left\{ \begin{aligned}
    m &= 6 \\
    \FF &= \{123, 234, 124, 134, 125, 235\} \\
    \partial \FF &= \{12, 23, 13, 24, 34, 14, 15, 25, 35\}
  \end{aligned} \right.$

\end{itemize}

\begin{obs}
  Az előbbi példa alapján kézenfekvőnek tűnik az a sejtés, hogy a $\binom{[n]}{r}$ halmazoknak létezik olyan sorrendje, hogy a sorrend első $m$ elemének az árnyéka minimális. A későbbiekben látni fogjuk, hogy ez a sejtés igaz.
\end{obs}


A $\binom{[n]}{r}$ nevezetes sorrendjei közül kettő:
\begin{itemize}
  \item lexikografikus:\hphantom{ko} $\{a_1, a_2, ..., a_r\} < \{b_1, b_2, ..., b_r\}$, ha $\exists i: a_i < b_i$ és $\forall j < i: a_j = b_j$.

  Ez egy könnyen érhető, gyakorlatban sokszor használt rendezési módszer (pl telefonkönyv). Viszont nyilvánvalóan nem minimalizálja az halmazrendszer árnyékát. Az előző példában szereplő paraméterek esetén $m=3$-ra a lexikografikus sorozat első három elemének, az $\{123, 124, 125\}$ halmaznak már hét elemű az árnyéka, pedig láttuk, hogy lehetséges hat elemű árnyékot generálni.
  \item kolexikografikus: $\{a_1, a_2, ..., a_r\} < \{b_1, b_2, ..., b_r\}$, ha $\exists i: a_i < b_i$ és $\forall j > i: a_j = b_j$.

  Ez egy nagyon hasonló sorozatnak tűnik, az alapján, hogy a definícióban csak egyetlen relációjel iránya különbözik. Pedig valójában ez a sorozat igen eltérően viselkedik a lexikografikushoz képest. Ennek a megértéséhez először nézzük a három elemű kolexikografikus sorrend első pár elemét:

  \[\{123, 124, 134, 234, 125, 135, 235, 145, 245, 345, 126, 136, 236, 146, 246, 346, 156, 256, 356, 456\}\]

 A következő elem generálásához mindig a legbaloldalibb ``számjegyet'' próbáljuk meg növelni, úgy hogy az még kisebb maradjon az utána következő számjegynél. Illetve ha egy számjegyet megnöveltünk akkor a tőle balra lévőket visszaállítjuk a lehető legkisebb értékükre.

 Például az 135-ben a legelső számjegy növelhető, és a legelsőtől balra nincs senki ezért a következő elem a 235 lesz. A 235-ben az első ``számjegy'' már nem növelhető, de a második igen, ezért az megnöveljük négyre, és az első elemet visszaállítjuk egyesre, tehát 145 lesz a következő. Aztán amire eljutottunk a 345-höz már felsoroltuk az $\binom{[5]}{3}$ összes halmazát, ezért be kellett vennünk a 6-os nevű elemet (az utolsó, legnagyobb ``számjegyet'' kell megnövelnünk), hogy új halmazt kapjunk.
\end{itemize}
