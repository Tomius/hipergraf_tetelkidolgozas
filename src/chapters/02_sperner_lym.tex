\chapter{Sperner tétel és LYM egyenlőtlenség}

\begin{dfn}
  Egy $\AA \subseteq 2^{[n]}$ \emph{Sperner rendszer}, ha $\forall A, B \in \AA: A \not \subseteq B, B \not \subseteq A$, ha $A \neq B$.
\end{dfn}

\begin{thm} Sperner-tétel:
  Ha $\AA$ egy Sperner rendszer, akkor $|\AA| \leq \binom{n}{\lfloor \frac{n}{2} \rfloor}$
\end{thm}

Bizonyítás (dupla leszámlálással):

\begin{notation}
  $S_n$ jelentse az n elemű szimmetrikus csoportot, azaz az első n elem permutációját.
\end{notation}

Tekintsük az olyan $(A, \sigma)$ párokat, ahol $A \in \AA$ és $\sigma \in S_n$, $\sigma = \sigma(1)\sigma(2) \dots \sigma(n)$ és $A = \sigma(1)\sigma(2) \dots \sigma(k)$, valamely k-ra (k=|A|). Azaz az $A$ ``párja'' $\sigma$-nak azt jelenti, hogy az $A$ eleminek létezik olyan sorrendje, hogy az így kapott sorozat prefixe a $\sigma$-nak. Az ilyen párok halmazát jelöljük $T$-vel.

\medskip

Egyrészt rögzített $\sigma \in S_n$-re legfeljebb egy $A \in \AA$ lehet, ami az ő párja, mert ha több is lenne, akkor azok tartalmaznák egymást, ami a Sperner-rendszer feltétele miatt ellentmondás. Emiatt $T$-nek legfeljebb annyi eleme lehet, ahány különböző értéket a $\sigma$ felvehet:
\[|T| \leq |S_n| = n!.\]

Másrészt rögzített $A \in \AA$ esetén a hozzá párba állítható $\sigma$-k számát csak az $A$-beli elemek sorrendje, és az $A$-ban nem szereplő, a $\sigma$ első $|A|$ eleme után folytatódó elemek sorrendje befolyásolja, vagyis a T-beli elemek száma:
\[|T| = \sum_{A \in \AA} |A|! (n-|A|)! \geq |\AA| \lfloor \frac{n}{2} \rfloor ! \lceil \frac{n}{2} \rceil !.\]

T elemszámára két összefüggést kaptunk, ezeket összevetve:
\begin{align}
  |\AA| \lfloor \frac{n}{2} \rfloor ! \lceil \frac{n}{2} \rceil ! &\leq |T| \leq n! \\
  |\AA| &\leq \frac{n!}{\lfloor \frac{n}{2} \rfloor ! \lceil \frac{n}{2} \rceil !}\\
  |\AA| &\leq \binom{n}{\lfloor \frac{n}{2} \rfloor}.
\end{align}

\QED

\begin{thm} LYM-tétel:
  Ha $\AA \subseteq 2^{[n]}$ Sperner rendszer és $f_k$ jelöli minden $k$-ra az $\AA$-ban lévő pontosan $k$ elemű halmazok számát, akkor $\sum_{k=0}^{n} \frac{f_k}{\binom{n}{k}} \leq 1$
\end{thm}

Biz (előző alapján):

\[\sum_{A \in \AA} |A|! \cdot (n - |A|)! \leq n!\]

\[\sum_{A \in \AA} \frac{1}{\frac{n!}{|A|! \cdot (n - |A|)!}} \leq 1\]

\[\sum_{A \in \AA} \frac{1}{\binom{n}{|A|}} \leq 1\]

\[\sum_{k = 0}^{n} \frac{f_k}{\binom{n}{k}} \leq 1\]

\QED

\begin{dfn} Lánc:
  $L \subseteq 2^{[n]}$, amire $B, B' \in L \Rightarrow B \subseteq B'$ vagy $B' \subseteq B$
\end{dfn}

Sperner-tétel másik bizonyítása:
Azt mutatjuk meg, hogy $2^{[n]}$ particionálható $\binom{n}{\lfloor \frac{n}{2}\rfloor}$ láncra.

Definiáljuk $\binom{[n]}{k}$ és $\binom{[n]}{k-1}$ között azt a $\GG$ páros gráfot, ahol az élek tartalmazást jelentenek: $A \in \binom{[n]}{k}$ és $B \in \binom{[n]}{k-1}$-re $AB$ él, pontosan akkor, ha $B \subset A$.

\begin{prop}
  Ha $k > \frac{n}{2}$ akkor $\GG$-ben $\exists \binom{[n]}{k}$-t fedő párosítás, ha pedig $k \leq \frac{n}{2}$ akkor $\exists \binom{[n]}{k-1}$-et fedő párosítás.
\end{prop}

Állítás biz ($k > \frac{n}{2}$ eset):
Legyen $x \subseteq \binom{[n]}{k}$. $\forall A \in x$-nek pontosan $k$ szomszédja van $\binom{[n]}{k-1}$-ben. Tehát $x$ és $N(x)$ között $|X| \cdot k$ él fut. $\forall B \in N(x)$-nek $n-k+1$ szomszédja van $\binom{[n]}{k}$-ban (ezek lehetnek $x$-ben és $x$-en kívül is) $\Rightarrow$ Az $x$ és $N(x)$ közötti élek száma $\leq |N(x)| \cdot (n-k+1)$.

Tehát $|x| \cdot k \leq |N(x)|(n-k+1)$, azaz $|N(x)| \geq |x| \frac{k}{n-k-1} \geq |x|$, ha $k > \frac{n}{2}$, mert $k > \frac{n}{2}$-re $k \geq n-k+1$.

\medskip

Állítás biz ($k \geq \frac{n}{2}$ eset):
Hasonlóan, $|x| (n-k+1) \leq |N(x)| \cdot k$, azaz $|N(x)| \geq |x| \frac{n-k+1}{k} \geq |x|$, mert $k \geq \frac{n}{2}$-re $n-k+1 \geq k$.

\medskip

A kapott párosítások felhasználásával adódik a láncokra való felbontás. Mivel egy Sperner rendszerben nem lehet több elem egyetlen láncbók, ha $\AA$ Sperner rendszer, akkor $|\AA| \leq \binom{n}{\lfloor \frac{n}{2} \rfloor}$, azaz a láncfelbontásbeli láncok száma.

\QED
