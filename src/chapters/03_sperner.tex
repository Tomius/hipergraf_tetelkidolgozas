\chapter{A Sperner rendszerek}

A Sperner rendszerek létjogosultságának megértéséhez először tekintsük az alábbi OKTV döntő feladatot 1979-ből:

\begin{task}
Adott 70 tolmács, akik közül bármely két $A, B$ tolmácsra igaz, hogy $A$ beszél olyan nyelvet, amit $B$ nem. Legkevesebb hány különböző nyelvet beszél a 70 tolmács összesen?
\end{task}

Empirikus próbálgatással könnyen észrevehetjük, hogy 8 nyelv elég, ugyanis minden tolmácshoz négy különböző nyelvet pontosan $\binom{8}{4} = 70$ féleképpen lehet rendelni.

A Sperner-tétel bizonyítást is ad arra, hogy 8 nyelvet ennél ``okosabban'' nem tudunk szétosztani tolmácsok között, azaz 8 nyelv esetén legfeljebb 70 darab, a feltételnek megfelelő tolmács lehet. Ehhez persze először szükségünk van a probléma formalizációjára, ahol a tolmácsokra mint nyelvek halmazára tekintünk, az összes tolmács egy halmazrendszert alkot, és ennek az elemszámára szeretnénk valamilyen állítást.

\begin{dfn}
Egy $\AA \subseteq 2^{[n]}$ \emph{Sperner rendszer}, ha $\forall A, B \in \AA: A \not \subseteq B, B \not \subseteq A$, ha $A \neq B$.
\end{dfn}

\begin{thm} Sperner-tétel:
Ha $\AA$ egy Sperner rendszer, akkor $|\AA| \leq \binom{n}{\lfloor \frac{n}{2} \rfloor}$
\end{thm}

Bizonyítás (dupla leszámlálással):

(Jelölés: $S_n$ jelentse az n elemű szimmetrikus csoportot, azaz az első n elem permutációját.)

Tekintsük az olyan $(A, \sigma)$ párokat, ahol $A \in \AA$ és $\sigma \in S_n$, $\sigma = \sigma(1)\sigma(2) \dots \sigma(n)$ és $A = \sigma(1)\sigma(2) \dots \sigma(k)$, valamely k-ra (k=|A|). Azaz az $A$ ``párja'' $\sigma$-nak azt jelenti, hogy az $A$ eleminek létezik olyan sorrendje, hogy az így kapott sorozat prefixe a $\sigma$-nak. Az ilyen párok halmazát jelöljük $T$-vel.

Egyrészt rögzített $\sigma \in S_n$-re legfeljebb egy $A \in \AA$ lehet, ami az ő párja, mert ha több is lenne, akkor azok tartalmaznák egymást, ami a Sperner-rendszer feltétele miatt ellentmondás. Emiatt $T$-nek legfeljebb annyi eleme lehet, ahány különböző értéket a $\sigma$ felvehet:
\[|T| \leq |S_n| = n!.\]

Másrészt rögzített $A \in \AA$ esetén a hozzá párba állítható $\sigma$-k számát csak az $A$-beli elemek sorrendje, és az $A$-ban nem szereplő, a $\sigma$ első $|A|$ eleme után folytatódó elemek sorrendje befolyásolja, vagyis a T-beli elemek száma:
\[|T| = \sum_{A \in \AA} |A|! (n-|A|)! \geq |\AA| \lfloor \frac{n}{2} \rfloor ! \lceil \frac{n}{2} \rceil !.\]

T elemszámára két összefüggést kaptunk, ezeket összevetve:
\begin{align}
|\AA| \lfloor \frac{n}{2} \rfloor ! \lceil \frac{n}{2} \rceil ! &\leq |T| \leq n! \\
|\AA| &\leq \frac{n!}{\lfloor \frac{n}{2} \rfloor ! \lceil \frac{n}{2} \rceil !}\\
|\AA| &\leq \binom{n}{\lfloor \frac{n}{2} \rfloor}.
\end{align}

\QED

\begin{thm} LYM-tétel:
<TODO>
\end{thm}

Biz <TODO>

Sperner-tétel másik bizonyítása: <TODO>

\begin{dfn} Részben rendezett halmaz:
<TODO>
\end{dfn}

\begin{thm} Dilworth tétele:
<TODO>
\end{thm}

\begin{dfn} Összehasonlítási gráf:
<TODO>
\end{dfn}

\begin{dfn} Perfekt gráf:
<TODO>
\end{dfn}

\begin{thm} Perfekt gráf tétel:
<TODO>
\end{thm}

\begin{thm}
  Minden összehasonlítási gráf egyben perfekt is.
\end{thm}

Biz: <TODO>

\begin{thm} Bollobás egyenlőtlenség:
  <TODO>
\end{thm}

Biz: <TODO>

\begin{thm} Ahlswedc-Zhang azonosság:
  <TODO>
\end{thm}

Biz: <TODO>
